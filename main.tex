\documentclass{article}
\usepackage[utf8]{inputenc}
\usepackage[spanish]{babel}
\usepackage{listings}
\usepackage{graphicx}
\graphicspath{ {images/} }
\usepackage{cite}

\begin{document}

\begin{titlepage}
    \begin{center}
        \vspace*{1cm}
            
        \Huge
        \textbf{Taller Calistenia}
            
        \vspace{0.5cm}
        \LARGE
        Informatica II
            
        \vspace{1.5cm}
            
        \textbf{Jorge Andrés Montaña Cisneros}
            
        \vfill
            
        \vspace{0.8cm}
            
        \Large
        Despartamento de Ingeniería Electrónica y Telecomunicaciones\\
        Universidad de Antioquia\\
        Medellín\\
        Marzo de 2021
            
    \end{center}
\end{titlepage}

\tableofcontents

\newpage

\section{Pasos para llevar objetos de una posición A a una posición B}

1.	Los objetos deben estar en una posición inicial, las tarjetas debajo de una hoja de papel, todo sobre una superficie plana horizontal, ejemplo una mesa o escritorio.\\

2.	En primera instancia se debe levantar la hoja con una sola mano y colocar está a pocos centímetros de las tarjetas de tal modo que la hoja no cubra las tarjetas y en lo posible sobre la misma superficie plana horizontal donde se empezó el ejercicio.\\


3.	Una vez realizado el paso anterior, levantar las dos tarjetas de la superficie horizontal, esta acción se debe realizar con una sola mano, en lo posible con la mano dominate, es decir la mano con la que se sienta más cómodo para realizar tareas regulares; una vez se tengan las tarjetas en la mano intente colocarlas de manera vertical, de tal manera que uno  de los bordes de menor longitud  quede mirando al piso, después ubique los dedos de la siguiente manera, el dedo meñique en el borde inferior de la tarjeta, es decir el lado más corto que está mirando hacia el piso, los dedos anular, pulgar y el del medio en los dos extremos o bordes más largos de la tarjeta, y por último el dedo índice en el borde superior, todo esto es con el fin de  emparéjelas, es decir que las dos tarjetas queden de manera uniforme.\\


4.	Una vez tengamos las tarjetas ordenadas en nuestra mano dominante procedemos ubicarnos encima de la hoja blanca que retiramos anteriormente, recuerde que las tarjetas deben de estar en posición vertical. Una vez asegurado de que estén en posición correcta, procedemos a colocar las tarjetas en la hoja de papel, para esto retiramos el dedo meñique del borde inferior de la tarjeta para poder  apoyarlo sobre la hoja de papel sin soltar los  dedos restantes, los cuales sostienen las tarjetas para que no se caigan.\\ 


5.	Una vez tenga las tarjetas encima de la hoja de papel, procedemos a separar las tarjetas de tal manera que queden en equilibrio y formen una pirámide,  para esto se deberán utilizar los dedos como ayuda; con el dedo índice podremos hacer presión en el borde superior  para mantener las tarjetas firmes, los dedos que están ubicados en los extremos laterales de la tarjeta, los usaremos para ir separándolas poco a poco hasta conseguir la forma deseada (Piramide), luego tratar de retirar los dedos lentamente y con cuidado de que no pierda el equilibrio es decir que mantenga forma. Si se caen, repetir el proceso desde el paso 3 hasta lograr conseguirlo. 


\end{document}

